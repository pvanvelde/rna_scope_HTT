\caption{\textbf{Empirical definition of the single-molecule regime via per-slide normalization and breakpoint analysis.}
\textbf{(A--B)} Per-slide intensity normalization.
\textbf{(A)} Raw integrated photon counts for each slide (faint colored lines, kernel density estimates). Substantial slide-to-slide variation in absolute photon yield is evident (peaks vary by $2$--$3\times$), arising from technical factors including tissue autofluorescence, fixation quality, probe hybridization efficiency, and imaging conditions.
\textbf{(B)} After per-slide normalization, setting the modal intensity to $N_{1\mathrm{mRNA}} = 1$ for each slide independently. The red dashed line marks $N_{1\mathrm{mRNA}} = 1$. All slide distributions now converge around this reference value, demonstrating successful removal of slide-to-slide technical variation while preserving the biological signal (tail extending beyond 1, representing multi-transcript aggregates).
%
\textbf{(C--E)} Biphasic relationships between normalized intensity and spot size.
For each dimension---\textbf{(C)} $\sigma_x$ (lateral $x$), \textbf{(D)} $\sigma_y$ (lateral $y$), \textbf{(E)} $\sigma_z$ (axial $z$)---the bold colored line shows the mean normalized intensity computed from all 1,816,872 spots pooled across 26 slides, using 5~nm bins (range: 80\% of PSF to 98th percentile of data).
The light shaded region ($\pm 1$ standard deviation across per-slide means) represents inter-slide variability in the biphasic relationship, while the narrower darker error bands show $\pm 1$ SEM (standard error of the mean) of the combined mean.
Purple dotted vertical lines mark the bead-derived PSF from fluorescent microsphere calibration ($\sigma_x = 185$~nm, $\sigma_y = 187$~nm, $\sigma_z = 573$~nm), representing the diffraction-limited reference.
Colored dashed vertical lines indicate empirically-determined breakpoints ($\sigma_x = 316$~nm, $\sigma_y = 317$~nm, $\sigma_z = 699$~nm), which are $71\%$, $70\%$, and $22\%$ larger than bead PSF values, respectively.
%
The biphasic pattern is diagnostic:
\emph{Left of breakpoint} (single-molecule regime): Intensity increases approximately linearly with size, consistent with single diffraction-limited emitters where larger fitted widths arise from increased local probe density or extended mRNA hybridization sites.
\emph{Right of breakpoint} (aggregate regime): Intensity plateaus despite continued size growth, indicating spatially unresolved multi-transcript clusters where additional size reflects aggregation rather than increased probe binding.
Breakpoints were determined via weighted piecewise-linear regression with Savitzky--Golay smoothing (window = 9 bins, polynomial order = 2), searching only after intensity reaches 80\% of maximum and applying strict penalties to enforce the expected biphasic structure (positive slope before breakpoint, near-zero slope after).
%
\textbf{(F--H)} Spot size distributions.
For each dimension---\textbf{(F)} $\sigma_x$, \textbf{(G)} $\sigma_y$, \textbf{(H)} $\sigma_z$---faint colored lines show per-slide probability density functions (kernel density estimates with Scott's bandwidth), while the thick colored line shows the combined PDF from all spots pooled across slides.
Purple dotted lines mark the bead PSF, and colored dashed lines mark the modes (peaks) of the combined distributions, which define the tissue-calibrated refined PSF: $\sigma_x = 259$~nm, $\sigma_y = 258$~nm, $\sigma_z = 669$~nm.
These refined PSF values supersede the bead-derived measurements for all downstream single-molecule quantification, as they account for tissue-specific effects (refractive index mismatch, optical aberrations, finite probe cluster size).
%
\textbf{Key findings:}
(1) Per-slide normalization successfully removes technical variation (Panel~B) while preserving biological heterogeneity.
(2) Empirically-determined breakpoints exceed bead PSF by $71$--$22\%$, consistent with the physical extent of RNAscope probe clusters ($\sim$20 probe pairs, each $\sim$2--3~nm) bound to target mRNA.
(3) The biphasic intensity--size relationship provides an objective, data-driven criterion for distinguishing single molecules (positive slope regime) from aggregates (plateau regime).
(4) Tissue-calibrated modes from size distributions (Panels~F--H) provide refined PSF estimates that account for \emph{in situ} optical conditions.
%
\textbf{Methods:}
Analysis performed on 1,816,872 spots across 26 slides from Q111 transgenic mouse tissue, using RNAscope \emph{in situ} hybridization with fluorescent detection at 488~nm excitation (green channel).
Quality filtering: probability of false alarm $< 10^{-4}$.
Per-slide normalization: kernel density estimation (KDE) with Scott's rule to identify modal intensity, then division by peak to set $N_{1\mathrm{mRNA}} = 1$.
Spot size determination: 3D Gaussian fitting to extract $\sigma_x$, $\sigma_y$, $\sigma_z$.
Binning: 50 bins spanning [\num{148.0}, \num{650.0}]~nm (lateral) and [\num{458.40000000000003}, \num{1278.0111059570406}]~nm (axial).
Imaging parameters: pixel size = 108.0~nm (lateral), slice depth = 200.0~nm (axial).
Statistical analysis: Python 3.x with \texttt{scipy}, \texttt{numpy}, \texttt{matplotlib}.
}