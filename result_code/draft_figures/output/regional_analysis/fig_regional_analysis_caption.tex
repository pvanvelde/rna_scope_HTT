Figure: Regional Analysis - Sub-regional Homogeneity vs Inter-regional Differences

This figure presents a comprehensive analysis of fl-HTT mRNA expression patterns across brain regions,
demonstrating that while sub-regional variation within Cortex and Striatum is minimal (location within
a region doesn't significantly affect expression), substantial differences exist between the two major
brain regions.

DATA FILTERING AND QUALITY CONTROL:
- Dataset: Q111 experimental samples from the ExperimentalQ111 - 488mHT - 548mHTa - 647Darp probe set
- Excluded slides (n=14): m1a1, m1a2, m1b2, m1b3, m1b4, m1b5, m2a1, m2a7, m2b2, m2b4, m2b5, m2b5, m3b4, m3b5
  (Slides excluded due to poor UBC positive control expression indicating technical failures)
- CV threshold for cluster filtering: CV >= 0.5
- Minimum nuclei per FOV: 40.0
- Intensity threshold: Per-slide, determined from negative control at quantile=0.95, max PFA=0.05

DATA SUMMARY:
- Slides analyzed (n=15): m1a4, m1a5, m1b1, m2a2, m2a3, m2a4, m2a8, m2b1, m2b7, m3a1, m3a2, m3a3, m3a5, m3b2, m3b3
- Total FOVs: 2668
  - Cortex: 1394 FOVs
  - Striatum: 1274 FOVs
- Age range: 2 - 12 months
- Brain atlas coordinate range: 40.00 - 68.00 mm

SUBREGIONS ANALYZED:
- Cortex subregions (n=5): Cortex - Piriform area, Cortex - Primary and secondary motor areas, Cortex - Primary somatosensory (mouth, upper limb), Cortex - Supplemental/primary somatosensory (nose), Cortex - Visceral/gustatory/agranular areas
- Striatum subregions (n=4): Striatum - lower left, Striatum - lower right, Striatum - upper left, Striatum - upper right

VOXEL AND PIXEL PARAMETERS:
- Pixel size (XY): 162.5 nm
- Slice depth (Z): 500 nm
- Voxel size: 0.013203125000000001 $\mu$m$^3$
- Mean nuclear volume (for nuclei estimation): 716 $\mu$m$^3$

STATISTICAL RESULTS:

HTT1A:

  Cortex sub-regional ANOVA: F=0.586, p=0.6739 ns
  Striatum sub-regional ANOVA: F=0.135, p=0.9385 ns
  Cortex vs Striatum (paired t-test):
    Cortex: 14.985 ± 14.051 mRNA/cell
    Striatum: 15.992 ± 16.096 mRNA/cell
    t=-0.589, p=0.5657 ns (n=14 paired samples)

FL-HTT:

  Cortex sub-regional ANOVA: F=0.554, p=0.6969 ns
  Striatum sub-regional ANOVA: F=0.030, p=0.9929 ns
  Cortex vs Striatum (paired t-test):
    Cortex: 25.237 ± 22.066 mRNA/cell
    Striatum: 22.003 ± 17.156 mRNA/cell
    t=1.709, p=0.1113 ns (n=14 paired samples)

PANEL DESCRIPTIONS:

Row 1 - Cortex Sub-regional Analysis:
A. HTT1a expression across cortical subregions with one-way ANOVA
B. fl-fl-HTT expression across cortical subregions with one-way ANOVA

Row 2 - Striatum Sub-regional Analysis:
C. HTT1a expression across striatal subregions with one-way ANOVA
D. fl-fl-HTT expression across striatal subregions with one-way ANOVA

Row 3 - Inter-regional Comparison:
E. HTT1a Cortex vs Striatum with paired t-test and connecting lines
F. fl-HTT Cortex vs Striatum with paired t-test and connecting lines

INTERPRETATION:
- Non-significant sub-regional ANOVAs indicate expression is homogeneous within each major brain region
- This validates pooling FOVs across subregions for region-level analyses
- Significant Cortex vs Striatum differences reflect true regional variation in fl-HTT expression
- Paired t-tests control for inter-animal variability by comparing regions within the same slide

METHODOLOGY:
- Sub-regional analysis: One-way ANOVA with Tukey's HSD post-hoc test ($\alpha$=0.05)
- Inter-regional analysis: Paired t-test using slide-level averages
- Expression normalized: (N\_spots + I\_cluster\_total / I\_single\_peak) / N\_nuclei
- Peak intensity: Slide-specific KDE mode estimation from single spots

Analysis performed with scienceplots style. CV threshold=0.5, Min nuclei=40.0.