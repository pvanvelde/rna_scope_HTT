\caption{\textbf{Empirical definition of the single-molecule regime in tissue via per-slide normalization and breakpoint analysis (Orange channel (548 nm), fl-HTT (complete mutant huntingtin transcript)).}
\textbf{(A)} Raw integrated photon counts for each of 26 slides analyzed (faint colored lines show kernel density estimates for individual slides, 4,474,897 total spots from Q111 transgenic mouse tissue).
Substantial slide-to-slide variation is evident, with peak intensities varying by $2$--$3\times$ due to technical factors: tissue autofluorescence, fixation quality, probe hybridization efficiency, and imaging conditions (laser power, detector sensitivity).
This heterogeneity necessitates per-slide normalization rather than global intensity calibration.
\textbf{(B)} After per-slide normalization, all distributions converge around $N_{1\mathrm{mRNA}} = 1$ (red dashed line), where the modal intensity of each slide is independently set to unity via kernel density estimation (KDE with Scott's bandwidth).
This normalization successfully removes technical variation while preserving biological signal, as evidenced by the conserved tail extending beyond 1 (representing multi-transcript aggregates and clusters).
%
\textbf{(C--E)} Biphasic relationships reveal the empirical boundary between single molecules and aggregates.
For \textbf{(C)} lateral $\sigma_x$, \textbf{(D)} lateral $\sigma_y$, and \textbf{(E)} axial $\sigma_z$, the bold colored line shows mean normalized intensity from all pooled spots using 8.0~nm bins.
The analysis range uses a principled hybrid approach: for lateral dimensions, the lower bound is 50\% of bead PSF (since spots cannot be smaller than approximately half the diffraction limit); for the axial dimension, the lower bound is fixed at 500~nm (1 slice depth) to capture the full rising phase of the biphasic curve. The upper bound is data-driven (98th percentile of data, adapting to actual distribution).
This yields ranges of [92, 494]~nm for $\sigma_x$/$\sigma_y$ and [500, 962]~nm for $\sigma_z$.
Light shading indicates $\pm 1$ standard deviation across per-slide means (inter-slide variability) and darker narrow bands show $\pm 1$ SEM of the combined mean (uncertainty).
Purple dotted vertical lines mark bead-derived PSF from fluorescent microsphere calibration ($\sigma_x = 185$~nm, $\sigma_y = 187$~nm, $\sigma_z = 573$~nm), representing the diffraction-limited optical reference.
Colored dashed vertical lines indicate empirically-determined breakpoints ($\sigma_x = 287$~nm, $\sigma_y = 287$~nm, $\sigma_z = 708$~nm), which exceed bead PSF by $55\%$, $54\%$, and $24\%$ respectively---consistent with the finite physical extent of RNAscope probe clusters ($\sim$20 probe pairs, each $\sim$2--3~nm) bound to target mRNA in tissue.
The diagnostic biphasic pattern separates two regimes: left of the breakpoint (single-molecule regime), intensity increases linearly with size, indicating that larger fitted widths arise from increased local probe density or extended hybridization sites while remaining single diffraction-limited emitters; right of the breakpoint (aggregate regime), intensity plateaus despite size growth, revealing spatially unresolved multi-transcript clusters where size reflects aggregation rather than probe binding.
Breakpoints were computed via weighted piecewise-linear regression with Savitzky--Golay smoothing (window = 9 bins, polynomial order = 2), searching only after intensity reaches 80\% of maximum and applying strict penalties to enforce positive slope before breakpoint and near-zero slope after (details in Methods).
%
\textbf{(F--H)} Spot size distributions define tissue-calibrated PSF values.
For \textbf{(F)} $\sigma_x$, \textbf{(G)} $\sigma_y$, and \textbf{(H)} $\sigma_z$, faint colored lines show per-slide probability density functions while thick colored lines show combined PDFs from all pooled spots.
Purple dotted lines replicate bead PSF references, while colored dashed lines mark the modes of the combined distributions: $\sigma_x = 248$~nm, $\sigma_y = 247$~nm, $\sigma_z = 659$~nm.
These mode values represent the most probable spot sizes in tissue and define the refined, tissue-calibrated PSF that supersedes bead-derived measurements for all downstream single-molecule quantification, accounting for \emph{in situ} effects including refractive index mismatch between tissue and immersion medium, optical aberrations from tissue heterogeneity, and the finite geometry of probe--mRNA complexes.
The close correspondence between modes and bead PSF ($<$5\% difference for $\sigma_z$, $\sim$40\% for lateral dimensions) validates the optical calibration while revealing tissue-specific broadening from probe cluster size.
Spot sizes were determined by 3D Gaussian fitting to background-subtracted fluorescence images acquired at 548~nm excitation (Orange channel (548 nm), detecting fl-HTT (complete mutant huntingtin transcript)) with 108.0~nm lateral pixel size and 200.0~nm axial slice depth.
Quality filtering retained spots with probability of false alarm $< 10^{-4}$.
}